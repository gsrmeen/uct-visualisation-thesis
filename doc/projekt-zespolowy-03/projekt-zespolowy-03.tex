\documentclass{article}
\usepackage{titling}
\usepackage[T1]{fontenc}
\usepackage[polish]{babel}
\usepackage[OT4]{fontenc} 
% Margins in document
\usepackage[left=1.5cm, right=1.5cm, top=3cm]{geometry}

% Avoid  colons before tables' empty captions and change caption
\usepackage{caption}
\captionsetup[table]{name=Tab.}
\captionsetup[figure]{name=Rys.}

% Don't know why, it starts from 2
\addtocounter{table}{-1}

% Rename tables' suffix
\renewcommand{\tablename}{Tab.}

% Graphicx setup
\usepackage{graphicx}
\graphicspath{{grafiki/}{../grafiki/}}

% No separator between items
\usepackage{enumitem}
\setlist{nolistsep}

% Pagebreak before every \section
\let\oldsection\section
\renewcommand\section{\clearpage\oldsection}

% Vhistory setup
\usepackage[owncaptions]{vhistory}
\renewcommand{\vhhistoryname}{Historia zmian}
\renewcommand{\vhversionname}{Wersja}
\renewcommand{\vhdatename}{Data}
\renewcommand{\vhauthorname}{Autor(zy)}
\renewcommand{\vhchangename}{Zmiany}

% Bigger padding in tabulars
\usepackage{array}
\setlength\extrarowheight{3pt}

% Itemize in tabulars (avoid big margins with minipage)
\newcommand{\tabbeditemize}[1]{
	\begin{minipage}[t]{0.4\textwidth}
		\begin{itemize}[topsep=0mm,partopsep=0mm,leftmargin=4mm]
			#1
		\end{itemize}
\end{minipage}}

% Code command
\usepackage{xcolor}
\definecolor{light-gray}{gray}{1}
\newcommand{\code}[1]{\colorbox{light-gray}{\texttt{#1}}}

% Modulename command
\newcommand{\modulename}[1]{\textit{#1}}

% Listings setup
\usepackage{listings}
\definecolor{codegreen}{rgb}{0,0.6,0}
\definecolor{codegray}{rgb}{0.5,0.5,0.5}
\definecolor{codepurple}{rgb}{0.58,0,0.82}
\definecolor{backcolour}{rgb}{0.95,0.95,0.92}
\lstdefinestyle{mystyle}{
	backgroundcolor=\color{backcolour},   
	commentstyle=\color{codegreen},
	keywordstyle=\color{magenta},
	numberstyle=\tiny\color{codegray},
	stringstyle=\color{codepurple},
	basicstyle=\ttfamily\footnotesize,
	breakatwhitespace=false,         
	breaklines=true,                 
	captionpos=b,                    
	keepspaces=true,                 
	numbers=left,                    
	numbersep=5pt,                  
	showspaces=false,                
	showstringspaces=false,
	showtabs=false,                  
	tabsize=2
}
\lstset{style=mystyle}

% DOCUMENT
\title{
	Wizualizacja drzewa stanów algorytmu UCT \\
	\large Dokumentacja powykonawcza}

\author{Patryk Fijałkowski \\ Grzegorz Kacprowicz}
\begin{document}
\begin{titlingpage}
	\maketitle
	\vspace{3cm}
	\begin{abstract}
		Poniższy dokument zawiera ogólny zarys projektu. Aplikacja ma w zamyśle pozwalać na oglądanie i dokładną analizę rozgrywki z komputerem w jedną z dwóch gier planszowych. Istnieje możliwość łatwego rozszerzenia o kolejne gry spełniające założenia wylistowane w dokumencie. Dokument przeprowadza czytelnika przez wszystkie moduły aplikacji - zaczynając od tego odpowiedzialnego za wizualizację. Pierwszy moduł, będący najistotniejszym, będzie opierał się na usprawnionej wersji algorytmu Walkera. Opisane są również moduły odpowiedzialne za logikę zaimplementowanych gier, implementację algorytmu oraz serializowanie generowanych drzew wraz ze schematami serializacji. \modulename{Aplikacja główna}, czyli ostatni opisywany moduł, jest modułem służącym do prezentacji działania poprzednich modułów. Przedstawiony jest również schemat interfejsu użytkownika, dokładnie opisujący najistotniejsze okna aplikacji. Ostatni rozdział dokumentu opisuje i uzasadnia technologie wybrane do stworzenia aplikacji.
	\end{abstract}
\end{titlingpage}

\begin{versionhistory}
	\vhEntry{1.0}{10.12.2019}{PF|GK}{stworzenie szkicu dokumentu}
\end{versionhistory}
\tableofcontents
	
	
\section{Wdrożenie}
\section{Instrukcja instalacji}
\section{Dokumentacja techniczna}
\section{Poradnik użytkownika}
\end{document}
