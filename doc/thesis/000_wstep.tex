\documentclass[uct_visualisation_thesis.tex]{subfiles}

Algorytm UCT, będący usprawnieniem metody MCTS, jest powszechnie stosowanym algorytmem w sztucznej inteligencji. Analizuje on obiecujące ruchy na podstawie generowanego drzewa, równoważąc eksploatację najbardziej obiecujących z eksploracją rzadko analizowanych decyzji. Każdemu wierzchołkowi drzewa odpowiada pewien stan rozgrywki, z którego algorytm rozgrywa losowe symulacje, rozszerzając potem drzewo o kolejne możliwe stany. Sposób, w jaki rozrasta się opisywane drzewo, jest kluczowy dla podejmowania przez algorytm obiecujących decyzji. \\

W celu umożliwienia głębszego zrozumienia idei opisywanego zagadnienia postanowiono stworzyć aplikację, która stanowiłaby wygodne narzędzie dla użytkownika zainteresowanego tą tematyką. Pozwala ona na wizualizację drzew stanów algorytmu UCT generowanych podczas rozgrywki w dwie przykładowe gry, co stanowi jednocześnie cel biznesowy projektu.\\

Istnieje wiele metod pozwalających na przejrzystą wizualizację struktur danych, jakimi są drzewa. Przykładami mogą być tak zwane drzewa ukorzenione (ang.\textit{rooted trees}), promieniowe (ang. \textit{radial trees}), czy też mapy drzew. Ze względu na chęć ukazania w sposób czytelny dla użytkownika poszczególnych ruchów graczy jako kolejne poziomy drzewa, w prezentowanym rozwiązaniu zdecydowano się użyć drzew ukorzenionych jako podstawę wizualizacji.