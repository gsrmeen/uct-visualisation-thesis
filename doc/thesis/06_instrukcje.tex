\documentclass[uct_visualisation_thesis.tex]{subfiles}
W celu wykorzystania możliwości, które daje prezentowany system, należy uruchomić go na komputerze, który spełnia wymienione poniżej wymagania sprzętowe.

\begin{enumerate}
	\item Procesor: Intel Core i5-3470 3.2 GHz / AMD FX-8350 4 GHz.
	\item Pamięć RAM: 8 GB.
	\item Karta graficzna: Nvidia GTX 660 2GB / AMD HD 7870 2 GB.
	\item Miejsce na dysku twardym: 150 MB.
	\item System operacyjny: Windows 10 / Ubuntu 16.04.
\end{enumerate}

W celu zainstalowania aplikacji na dysku lokalnym należy rozpakować jedno z dostarczonych archiwum. Prezentowany system został zbudowany w formie aplikacji przenośnej, więc wszystkie komponenty potrzebne do uruchomienia aplikacji są dostarczone razem z nią. W zależności od systemu operacyjnego urządzenia, należy wybrać odpowiednio:

\begin{enumerate}
	\item dla urządzeń z systemem Windows - \textit{UCTVisualisation-portable.zip},
	\item dla urządzeń z systemem Linux - \textit{UCTVisualisation-portable.tar}.
\end{enumerate}

Aplikacja znajduje się w rozpakowanym katalogu o nazwie \textit{UCTVisualisation} i aby ją włączyć, należy uruchomić odpowiedni plik w rozpakowanym katalogu. W zależności od systemu operacyjnego urządzenia, należy wybrać odpowiednio:

\begin{enumerate}
	\item  dla urządzeń z systemem Windows - \textit{UCT Visualisation.exe},
	\item dla urządzeń z systemem Linux - \textit{UCT Visualisation}.
\end{enumerate}
