\documentclass{article}
\usepackage{titling}
\usepackage[T1]{fontenc}
\usepackage[polish]{babel}
\usepackage[utf8]{inputenc}
% Margins in document
\usepackage[left=1.5cm, right=1.5cm, top=3cm]{geometry}

% Avoid  colons before tables' empty captions and change caption
\usepackage{caption}
\captionsetup[table]{name=Tab.}
\captionsetup[figure]{name=Rys.}

% Don't know why, it starts from 2
\addtocounter{table}{-1}

% Rename tables' suffix
\renewcommand{\tablename}{Tab.}

% Graphicx setup
\usepackage{graphicx}
\graphicspath{{grafiki/}{../grafiki/}}

% No separator between items
\usepackage{enumitem}
\setlist{nolistsep}

% Pagebreak before every \section
\let\oldsection\section
\renewcommand\section{\clearpage\oldsection}

% Vhistory setup
\usepackage[owncaptions]{vhistory}
\renewcommand{\vhhistoryname}{Historia zmian}
\renewcommand{\vhversionname}{Wersja}
\renewcommand{\vhdatename}{Data}
\renewcommand{\vhauthorname}{Autor(zy)}
\renewcommand{\vhchangename}{Zmiany}

% Bigger padding in tabulars
\usepackage{array}
\setlength\extrarowheight{3pt}

% Itemize in tabulars (avoid big margins with minipage)
\newcommand{\tabbeditemize}[1]{
	\begin{minipage}[t]{0.4\textwidth}
		\begin{itemize}[topsep=0mm,partopsep=0mm,leftmargin=4mm]
			#1
		\end{itemize}
\end{minipage}}


% DOCUMENT
\title{
	Wizualizacja drzewa stanów algorytmu UCT \\
	\large Plan projektu}

\author{Patryk Fijałkowski \\ Grzegorz Kacprowicz}
\begin{document}
	\begin{titlingpage}
		\maketitle
		\vspace{3cm}
		\begin{abstract}
			Poniższy dokument zawiera ogólny zarys projektu-aplikacji. Składa się on z opisu architektury systemu i poszczególnych komponentów wraz z pełnioną funkcją. Opisane są tu interfejsy skłądających się na ten projekt modułów, użyte biblioteki i komunikacja między nimi. Znajdują się tu też diagramy UML, które opisują klasy w projekcie i ich zachowanie względem siebie. Dokument zawiera również przykładowy interfejs graficzny dla użytkownika. Co więcej, opisane są użyte technologie i wymagania sprzętowe.
		\end{abstract}
	\end{titlingpage}

	\begin{versionhistory}
		\vhEntry{1.0}{3.11.2019}{PF|GK}{stworzenie pierwszej wersji dokumentu}
	\end{versionhistory}
	
	\tableofcontents
	
	\section{Architektura aplikacji}
	Aplikacja będzie podzielona na 5 oddzielnych modułów: algorytm, serializacja, wizualizacja, gra 1, gra 2... Ten odpowiedzialny za to, tamten za to... Takie rozwiązanie jest dobre bo...
	
	
	\section{Moduły aplikacji}
	Szczegółowy opis każdego z modułów w subsection...
	
	
	\section{Główne komponenty aplikacji}
	Najważniejsze komponenty:
	\begin{itemize}
		\item MonteCarloTreeSearch, BaseGameMove, BaseGameState
		\item BaseSerializator, BinarySerializator, CsvSerializator
		\item MonteCarloNode, MonteCarloNodeDetails, MonteCarloVisualisationDetails
	\end{itemize}
	
	\section{Interfejs użytkownika}
	Szkic głównego okna z opisem...
	\section{Wybrane technologie}
	
\end{document}